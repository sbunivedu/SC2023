\documentclass{article}

%\input{preamble}

%%\addbibresource{paper.bib}

\title{Supporting Low-Income, Talented Undergraduate Students in Engineering and Computing Sciences with Scholarships and Mentoring\footnote{\protect\input{copyright}}
}

\author{
Dulal C. Kar, Scott A. King, and Dugan Um\\
Texas A\&M University-Corpus Christi\\
Corpus Christi, TX 78412\\
\email{\{dulal.kar, scott.king, dugan.um\}@tamucc.edu}\\
}

\begin{document}
\maketitle

\begin{abstract}
The NSF S-STEM program supports low-income, talented students seeking an education and career in STEM fields. This works presents the results of an NSF S-STEM grant awarded to Texas A\&M University-Corpus Christi, a Hispanic Serving Institution, that recruited and supported 39 talented and financially needy undergraduate students including 23 students from underrepresented groups. Each student was mentored, guided, and supported with curricular and co-curricular activities for engineering and computer science majors and relieved from financial burden of paying tuition and other expenses with an amount of \$7,000 per year.  There were 17 community college transfer students and 22 high school seniors recruited altogether. Despite the support of the scholarship, seven students could not continue in the program since they could not maintain the GPA of 3.0 required to stay in the program. All seven dropouts were from the group of 22 recruited high school seniors, and none were from the group of the 17 community college transfer students. There were 11 students in the Hispanic students group. It was found that the overall GPA of the Hispanic students group went down near the end of their graduation while the overall GPA of the rest of the students improved slightly. In various support services and activities, midterm mentoring was found to be most helpful for students at-risk to continue in the scholarship program and the participation in undergraduate research was found to be beneficial for some students in securing employment in leading industry. The findings and results of this S-STEM scholarship project are important to prospective S-STEM PIs (Principal Investigators) who plan to apply for NSF S-STEM grants and recruit Hispanics, women, and/or community college transfer participants.
\end{abstract}

\section{Introduction}
The NSF S-STEM scholarship program supports talented, needy undergraduate students with scholarships in S-STEM disciplines. The S-STEM program has helped many needy students around the nation to complete their education in a STEM discipline and then eventually enter a STEM career to contribute to advancement and applications of science, technology, engineering, and mathematics \cite{james, wilson, madsen, cutright, adams, gonzalez, russomanno2010memphistep, tinto2}. Texas A\&M University-Corpus Christi (TAMUCC) as a Hispanic Serving Institution received an S-STEM grant from NSF to offer scholarships to talented and disadvantaged undergraduate students majoring in engineering or computer science. The objectives of the S-STEM project were to recruit talented, needy undergraduates for majors in engineering and computer science disciplines, particularly as many as possible from underrepresented groups such as Hispanics and women and support them with mentoring, co-curricular, and professional development activities. Specifically, we targeted to recruit students not only from high schools but also from community colleges since many needy students attend a community college to further their education. 

\textbf{Community College Transfer Students:} The ever-increasing cost of education in four-year degree institutions has made college education unaffordable for many low-income high-school graduates. Many of them have found low-cost community colleges to further their education beyond high school. The American Association of Community Colleges (AACC) reports that the annual average cost of tuition and fees for students in community colleges in the United States is about \$3,300. In contrast, the annual average cost of tuition and fees at state funded institutions is approximately \$9,000. Some have found community colleges as affordable pathways to higher education first by completing a two-year associate program at a community college with less cost in tuition and fees and then transferring to a four-year institution that gives credits earned in a community college \cite{hagedorn2012realistic, springer, cerna, slater2006lessons, summers2006preparing}. However, upon completion of a community college degree, the additional increased cost of attending a four-year degree institution for the remaining two or three years of education can discourage a vast majority of  economically disadvantaged students from pursuing a four year degree. Those who transfer from a community college to a four-year institution face severe financial hardship due to increased tuition and fees and often have to work long hours in jobs to survive financially. The lack of time for study can cause poor academic performance for many of these individuals which in turn causes many to drop out from the pursuit of a four-year degree. Another consequence is delayed graduation as few courses are taken or courses with poor performance are repeated. The major goal of our NSF sponsored S-STEM scholarship program was to recruit and support community college transfer students financially in their pursuit of higher education by alleviating their need to work in low-paying, long-hour jobs and facilitating them with necessary time for study to be successful in their coursework.

\textbf{Diversity and Recruitment of Hispanics:} TAMUCC is a Hispanic Serving Institution with an enrollment of over 10,000 students, of which about 48\% are Hispanic.  However, Hispanic student enrollment percentages do not match the region’s 60\% Hispanic population.  Nationally, Hispanics remain considerably underrepresented in engineering and computer science careers \cite{roy2019engineering}. In 2017, ASEE reports that among the engineering BS degrees awarded, only about 11.1\% were awarded to Hispanic students. As noted earlier, partly due to financial hardship, many Americans, and most Hispanics in Texas, choose to attend community college for their education beyond high school. However, a National Academies report found that students entering community colleges do not realize that they can obtain a four-year STEM degree such as in computer science or engineering by transferring to a four-year institution \cite{academies2010rising}. Earlier studies have indicated only about 20\% of college-qualified low-income students transfer and attain a bachelor’s degree. Hispanic students are the most likely to leave college without attaining a degree in large part due to financial burdens and cultural dislocation.  It is to be mentioned that Hispanics are less likely to accept loans and take on that debt, and instead, they choose to work \cite{varorta2007latino}. Similarly, another well-known underrepresented group in our recruitment target for the S-STEM program was women. It is well-known that enrollment of females in engineering and computer science is very low nationally. An important focus of our S-STEM scholarship project was to recruit as many talented students as possible from underrepresented groups, including Hispanic students and women.

Providing financial support to these students does not automatically guarantee their success in achieving a bachelor’s degree from a four-year institution.  There is a significant disparity between the number of underrepresented students who begin in the engineering and computer science disciplines and the number who complete their degrees eventually. Many Hispanic or Latino students are first generation college students who do not receive much motivational support from their family, who may not understand the struggles these students are going through \cite{dennis2005role}. These students experience a multitude of difficulties to adapt and integrate with the mainstream student body. Support services are needed to help them succeed in their pursuance of a four-year degree in STEM disciplines \cite{chemers, gershenfeld, yomtov2017peer, tinto2002learned, kalevitch2016building}. Therefore, in addition to the financial support, our S-STEM scholarship project provided the academic, professional development, and emotional supports necessary to help these academically talented students succeed.   

In this paper, we present our experience based on our NSF S-STEM grant that awarded \$7,000 per year as a scholarship to each participant, which covered tuition and fees to alleviate the financial burden. It is to be mentioned that the current average cost of in-state tuition and fees at our institution for an undergraduate student is \$9,292.  This S-STEM project recruited altogether 39 students, in which 17 students were community college transfer students and the remaining 22 students started their college education at the institution directly after graduation from high school. Among the 39 recruited students, 11 students were Hispanic, and 13 students were female.   The diversity of students in the S-STEM project provided us the opportunity to share our valuable experience with the S-STEM community, particularly in engineering and computing. The grant did not have any research requirement other than supporting participants with scholarships, supervision, and engaging them in voluntary learning and professional development activities.


\section{Recruitment Efforts}

To support recruitment of participants as well as to disseminate information about the S-STEM scholarship opportunity at our institution, a website with essential information needed to know by an applicant was maintained. The website provided information on the benefits of the program, eligibility requirements to apply, documents to be submitted with the application including: transcripts, an essay, and references. An application form was developed to receive information such as on the applicant’s academic background, ethnicity, and work experience. A reference form was developed to receive feedback from the references on the applicant’s skills, motivation, and even economic situation, if known. A brochure on the S-STEM opportunity was developed with requisite information such as the eligibility requirements, the scholarship amount, professional development opportunities, and support services. 

An outreach and recruitment institutional representative from TAMUCC visited local and regional high schools and community colleges giving talks, distributing flyers, and meeting high school counselors and community college officials. TAMUCC has yearly designated days (official preview days) to conduct recruitment events on the campus designed especially for prospective undergraduate students and friends and family members to learn about the student life at the campus. Flyers on the S-STEM scholarship opportunity were handed out to the attendees during those events. Also, during incoming freshmen mentoring sessions and orientation sessions flyers were distributed to the incoming freshmen in engineering and computer science programs. 

The institutional outreach and recruitment representative attended career fairs for high school students in Houston, Dallas, and Rio Grande Valley areas and distributed flyers on the S-STEM scholarship opportunity. In regional “Counselor Updates” organized in Dallas, Houston, and Rio Grande Valley areas, admission counselors of our institution provided high school and community college counselors information about the S-STEM scholarship opportunity. 
An application package would include the application form, a resume, transcripts, an essay, and three completed reference forms. A rubric with evaluations points on GPA, quality of the essay, standardized test score (SAT or ACT sore), work experience, and reference information was developed to evaluate applications by a selection committee. For each applicant, an overall total score was calculated by assigning a weight to each score summing all weighted scores. The rank of each applicant was determined based on the overall score calculated. The recruitment decision on the number of participants for a program was made in proportion to the number of students enrolled in each program, either in engineering or computer science. Besides, ranking, diversity was also considered while selecting participants for the S-STEM project. 

The original project duration was five years, which was extended to seven years with approval from NSF. Over the period of the project, there were 39 students altogether recruited in the program including: White – 25, Hispanics – 11, African American – 1, Asian – 1, and Unreported Ethnicity – 1. Each of the participants was either a U.S. citizen, permanent resident, national, or refugee, as required by NSF, and each of them had a GPA of at least 3.0 at the time of recruitment. By gender, there were 13 women among the participants, exactly one-third of the total 39 participants. As stated above, according to our recruitment plan, we recruited 17 community college transfer students. Each student was supported with a scholarship of \$7,000 per year for up to four years for a regular undergraduate. The support period was shorter for community college transfer students. 

\section{Support Services and Activities}
Earlier studies show that mentoring contributes to the retention and academic success of students \cite{cutright, gershenfeld, yomtov2017peer, slater2006lessons}.  The S-STEM participants received academic and career support using three major mentoring strategies:  1) Faculty Mentoring, 2) Academic Counseling, and 3) Career Counselling. A team of faculty and staff met regularly with the participants to promote academic success, retention, and progress to graduation and employment. It is to be noted that in compliance with NSF S-STEM scholarship program requirements, participation in any curricular, co-curricular, or professional activities for a student was voluntary.

At the beginning of each semester, we would organize a mentoring session or retreat. Attendance to the retreat was mandatory for all students. To handle scholarship offers to participants as well as to oblige them to fulfill the requirements of the scholarship award, a contract document on the scholarship award was used. The contract document includes some terms and requirements such as being a full-time student in computer science or engineering, maintaining satisfactory academic progress (GPA > 2.9), completing FAFSA, attending mandatory meetings, and attending mentoring sessions and seminars. During each fall semester retreat, each participant would sign the contract of the scholarship containing a clause in the contract regarding the GPA requirement to continue in the program and to participate voluntarily in many activities and take advantage of available resources in the campus such as for tutoring, academic counseling, and career counseling needed to succeed in the pursuit of the BS degree.  In each retreat, the participants would receive useful information from academic counselors, career counselors, financial aids office administrators, and peer participants.  

During the retreat, the participants were reminded about the mission of the NSF S-STEM program and its significance at the national level in advancement of science, engineering, and mathematics. In this context, the purpose, goals, objectives, and expectations of the S-STEM scholarship project at TAMUCC were made known to the participants. Requirements in terms of academic performance and participation in activities were informed.

\textbf{Focusing on Education:} Career prospects in engineering and computer science as well as strategies to succeed in college education were discussed in retreats. It was observed that many scholarship recipients continued working and falling in danger of loosing the scholarship. They were advised to concentrate on education and graduate on time, rather than working part-time to supplement income for living expenses and graduate later. Job prospects for students graduating in engineering and computer science were highlighted to raise the level of their motivation. They were reminded that the scholarship support paying their tuition and fees should relieve them from working long hours in jobs and concentrate on study and they should not seek any jobs to support themselves while seeking college education. They were informed about availability of various support services within the campus such as the programming assistance lab, the writing center, and the tutoring center, and encouraged to take advantage of these services to advance their education. 

\textbf{Balancing Course Load:} Advising students with a balance of intensive, non-intensive, hands-on, technical, and non-technical courses is critical to student success and hence to retention. The scholars were instructed how to choose courses based on technical, analytical, and hands-on contents and accordingly, enroll in courses each semester to achieve academic goals each time successfully. They were advised to form study groups to succeed in difficult courses, study objectively from presentation materials used in classes with questions in mind, and develop problem solving skills in engineering, computer science, and mathematics by working on example problems step by step. Helpful tips were provided how to do well in college, how to manage time for various activities, and how to make use of support facilities available within the college.

\textbf{Memberships in Professional Societies:} Joining professional societies is very helpful for professional development. The participants were encouraged to join professional societies and informed about the availability of fund to pay for their memberships of professional societies such as IEEE, ACM, and ASME.

\textbf{Undergraduate Research Opportunities:} Importance of undergraduate research was stressed, particularly for a better job and career prospect. All participants were informed about the research opportunities available on campus and were encouraged to contact the engineering or computer science faculty members to join their research teams. TAMUCC supports many STEM undergraduates each year for research and travel to conferences through an NSF sponsored LSAMP (Louis Stokes Alliances for Minority Participation) grant. The S-STEM participants were informed to apply for the research opportunity. The participants were introduced with NSF sponsored REU programs offered at different institutions around the nation. We provided references to the participants who applied for this opportunity. In addition, the participants were encouraged to apply to the NSF sponsored computer science REU program offered at our institution. We demonstrated to the S-STEM scholarship recipients how to access NSF websites for engineering and computer science summer REU programs, explore NSF-funded REU programs in various institutions including the ones in the state of Texas, and find out specific information and benefits. Benefits of these programs including research experience, professional growth opportunities, stipends, and allowances for travel, food, and housing were stressed. The S-STEM participants were shown how to apply online for various REU opportunities.  Several participants joined the REU program offered at our institution and some other institutions around the nation. 

\textbf{Career Services:} One of the career counselors of our institution would give a presentation at each retreat on various resources and services available at the institution. Participants were informed on how they can use online tools and resources available for career planning (CHOICE360, BLS.GOV, and OnetOnline.org), preparing professional documents, job hunting, preparing for graduate/professional school, and networking with professionals. Internship opportunities and benefits were discussed. The S-STEM scholars were informed about professional etiquette to be followed during job interviews and asked to participate in mock interviews for practice. They were invited to attend career fairs organized at our institution throughout the year. During retreats, other presentations by University Services Senior Personnel include presentations by a financial aid advisor of the financial aid office, presentations by academic advisers of the College of Science and Engineering, and presentations by the directors of student support services, tutoring, learning, and writing centers.

\textbf{Presentations by Peer Students:} Several scholars who participated in undergraduate research programs including NSF-sponsored REU programs and presented papers or posters in conferences also gave presentations during retreats. Similarly, some of the scholars who did summer internships in regional industry also gave presentations to the participants during mentoring sessions or retreats. Motivated by these presentations, many of them applied for REU and internship opportunities.

 Beyond retreats, the following support services were provided to the S-STEM participants on a regular basis or as needed.
 
\textbf{Faculty-Student Mentoring:} Mentor training is provided by our institution’s Office of Student Engagement and Success. Faculty mentors are informed at the beginning of each academic year how to access online and other resources available at the institution to successfully mentor students. Information on student support services and mentoring models is the primary focus of the training.  It is understood that the mentoring needs of a student vary on ethnicity and socio-economic background as well as the academic level of the student. Faculty members are informed and trained accordingly with information and approaches needed to successfully mentor students of diverse ethnic background and academic level. A separate group of faculty members is responsible to mentor freshman-level students, particularly in the computer science program. 

\textbf{General Counseling:} The S-STEM participants had access to individual counseling, personal skills training focused on helping students improve goal setting, memory and study skills, workshops on topics such as wellness, healthy relationships, and assertiveness, plus many other topics. A number of services and computer software programs are available for free via the University’s Transfer Center.  S-STEM participants were recommended to take advantage of such services that included career selection and skill enhancement in note-taking, test strategies, time management, stress management, financial management, and public speaking. 

\textbf{Professional Interactions:} As mentioned above, the S-STEM participants are informed about the benefits and the funding available to become members of one group of their choice, such as ACM Computer Science Club, American Society of Mechanical Engineers Student Chapter, and Association for Women in Science. Time to time, these student organizations would bring in their meetings industry professionals for seminars and short talks thus providing them opportunities to learn about career prospects in industry.

\textbf{Midterm Mentoring:} Midterm mentoring was found to be the most beneficial support service for the students who were at risk of losing the scholarship due to low GPA. After the initial mentoring session or retreat each semester, there was a mid-term mentoring/advising meeting with some selected at-risk students. In this effort, only the students who were at-risk with low mid-term grades were contacted. The project leadership would meet each student individually and check midterm grades reported by faculty in all courses. Students having trouble in some courses were advised to meet course instructors and attend tutorial services provided by the university as well as provided with helpful tips such as to form study groups and join online forums.  Occasionally, a course instructor was also contacted to setup a meeting with the affected student to receive helpful guidance to improve performance. 

The first step in midterm mentoring was to find out the reason behind poor performance in a course. Often, students would reveal their struggle in their personal lives, and they would need to be directed to receive general counseling. For directly course related situations, depending on the situation, they were advised to meet with the course instructor to find out how to improve in the course, form a study group to improve understanding of course material, seek help from teaching assistants in programming assignments, visit the tutoring center on math or physics assignments, and make appointments with the writing center to improve writing. In some cases, it was found that a student was working long hours which was the main cause of poor performance in multiple courses, and it was necessary to advise the student to attain some balance between work and study. A powerful argument was to appeal to them that if they could not spend enough time on their studies, they were putting their scholarship at risk and, also that they should look at their scholarships as if they were being “paid” to study, and they, therefore, needed to fulfill that requirement. It was found that working long hours was one of the primary reasons for poor performance for many participants. After mind-term mentoring, it was found that some students would reduce their work hours to make more time to improve their grades. In some cases, it was necessary to help students by developing a well-balanced schedule of work and study as well as allocating ample time in the schedule to study for improving grades in difficult courses.

The NSF S-STEM program collects data on each S-STEM project on activities. Table 1 summarizes data collected in our project on all 39 students who were recruited and supported through the S-STEM project. It is to be noted that attending a retreat or mentoring session was mandatory for all participants and hence the corresponding participation rate was 100\%, as shown in the table. The “Other” activities include many miscellaneous activities such as joining a club, playing sports, providing tutoring services, helping in church services, and so on.


%%\scriptsize
\begin{table}[h]
\scriptsize
\centering
\caption{Participation in Activities} 
\begin{tabular}{|l|c|c|c|c|c|c|l|}
  %\hline
  %\multicolumn{8}{|c|}{Table 1. Participation in Activities} \\
  \hline 
  \multirow{2}{*}{Item} & \multicolumn{6}{|c|}{Academic Year} & \multirow{2}{*}{Yearly} \\
  \cline{2-7}
   & 2015- & 2016- & 2017- & 2018- & 2019- & 2020- &  \\
     & 2016 & 2017 & 2018 & 2019 & 2020 & 2021 & Rate \\
   \hline
  No. of Participants & 7 & 14 & 13 & 30 & 24 & 16 & -  \\
  Part-Time Employment & 3 & 5 & 6 & 17 & 12 & 9 & 50\%\\
  Academic Support Services &	4 & 5 & 8 & 19 & 15 & 10 & 59\% \\
  Career Counseling & 2 & 2 & 7 & 12 & 9 & 8 & 38\% \\
  Community Building & 0 & 5 & 7 & 7 & 5& 2& 25\%  \\
  Field Trips	& 0	& 2	& 3	& 7	& 6	& 1	& 18\%  \\
  Internships	& 1	& 0	& 1	& 9	& 3	& 1	& 14\% \\
  Meetings/Conferences & 2 & 7 & 4 & 17 & 15 & 9 & 52\% \\
  Mentoring & 7 & 14 & 13 & 30 & 24 & 16 & 100\% \\
  Recruitment & 0	& 0	& 2	& 5	& 3	& 1	& 11\%  \\
  Research & 0 & 1 & 1 & 7 & 4 & 3 & 15\%  \\
  Seminars & 4 & 4 & 3 & 27 & 7 & 3 & 46\% \\
  Other & 4 & 5 & 8 & 17 & 15	& 5	& 52\% \\
  \hline
\end{tabular}
\end{table}
%%\normalsize

\section{Results and Discussion}
As mentioned above, of the 39 students in the program, there were 23 students (59\% of the total number of participants) altogether from underrepresented groups including women. In terms of gender, there were 13 women among the participants. Ethnic minorities include 11 Hispanics, one African American, one Asian, and one student of undeclared ethnicity. In all, 29 students completed the S-STEM program successfully, including all 17 community college transfer students. Among the ten students who could not or did not continue in the program, seven students could not maintain the required minimum GPA of 3.0, one student switched to a different STEM major, one student transferred to a different institution and continued in a STEM major, and one student dropped out from our institution after adversely affected by a natural disaster (hurricane).  Overall, the attrition rate is 18\% (7 out of 39) due to low GPA. Over the period of the grant, 35 students have graduated with a BS degree either in engineering or computer science or in a STEM discipline, i.e., the graduate rate is about 90\%, thereby increasing the pool and diversity of the nation’s workforce in these critical STEM fields.  
Table 2 shows overall average GPAs of different groups of students. We observe improvement in the academic performance of the community college transfer students as indicated in their overall final average GPA. In contrast, the performance of high school seniors recruited in the program shows a decline in overall final average GPA by 2\%. In terms of gender, almost no change in academic performance is observed. A significant decline of 3\% in overall average GPA in academic performance in the Hispanic ethnic group is found.

\begin{table}[h]
\footnotesize%\scriptsize
\centering
\caption{Academic Performance by Ethnicity, Gender, and Initial Level of Education} 
\begin{tabular}{|l|l|l|l|l|}
  %\hline
  %\multicolumn{5}{|c|}{Table 2. Academic Performance by Ethnicity, Gender, and Initial Level of Education  } \\
  \hline
  \multicolumn{2}{|c|}{ } & Initial Avg. & Final Avg. & Interpretation \\
  \multicolumn{2}{|c|}{ } & GPA & GPA & \\
  \hline
  \multirow{2}{*}{Ethnicity} & Hispanic & 3.74 &	3.63 & -3\% (decrease in GPA) \\
  \cline{2-5}
  & White & 3.53  & 3.54 & No significant change \\
  \hline
  \multirow{2}{*}{Gender} & Female & 3.65	& 3.66	& No significant change \\
    \cline{2-5}
  & Male	& 3.55	& 3.53	& No significant change \\
  \hline
  \multirow{2}{*}{Group} & High School Senior & 3.69	& 3.61	& -2\% (decrease in GPA) \\
    \cline{2-5}
  & Community College & 3.54 & 3.60 & 2\% (increase in GPA) \\
  & Transfer & & & in GPA \\
  \hline
\end{tabular}
\end{table}

In the following, we summarize our findings and make suggestions that can be helpful for all participants, especially low-income, talented participants of Hispanic descent: 
\begin{enumerate}
\item \textbf{Scholarship Amount:} An amount of \$7,000 per year was not enough as a scholarship for most students as they had to work long hours to support themselves. Some students were found to hold jobs for 30 hours per week. Increasing scholarship amount will help these students to devote more time to study, continue in the program, and graduate on time. The maximum scholarship amount of \$10,000 is allowed by the NSF S-STEM program. It is recommended that the maximum amount should be given as a scholarship to each participant.  
\item \textbf{Midterm Mentoring:} Midterm mentoring was very effective in retaining the students in the S-STEM project, particularly the students at risk. Most students reported with midterm grades of C or D in courses were eventually able to attain better grades or to maintain the overall GPA of at least 3.0. More interventions such as midterm mentoring were found to be beneficial to improve academic performance of the participating students. For example, in Fall 2018, among 29 students, 17 students had a grade of C or D in one or more courses. These students were given guidance how to improve their situation in those courses. All these students did well and were able continue their scholarships in spring 2019 with an overall GPA greater than 2.9. Similarly, in spring 2019, among 25 students, eight students had a grade of C or D in one or more courses. Except one student, all other seven students at-risk were able to maintain a GPA above 2.9.
\item \textbf{Research Participation:} Several students involved in undergraduate research, including in NSF REU programs, were found to receive employment in leading industry or seek graduate education.  In surveys, many participants expressed positively on the impact of research experience in their academic pursuit, employment success, or career path.  
\item \textbf{Learning Community:} The dropout rate of high school senior recruits from the S-STEM program was found to be too high. One possible solution is to enlist them in learning communities/cohorts and guiding them with advice more frequently \cite{tinto2002learned, kalevitch2016building}. 
\item \textbf{Professional Development:} In compliance with NSF’s requirements, participations in co-curricular activities were voluntary for students. Accordingly, the students were advised to attend seminars, symposiums, and conferences as well as to join internships and undergraduate research programs. However, a focus group study reveals that many participants wanted to have more engaging professional development activities through workshops such as on Presentation Skills, Technical Writing, Resume Preparation, and Ethics. Due to lack of funding in the S-STEM grant to support such activities at the time, it was not possible for us to organize sessions or workshops. 
\end{enumerate}

\textbf{Focus Group Study:} A focus group study was conducted by an external evaluator. The focus group participants were predominantly sophomores (5) and juniors (4). One female senior participated. There were six male informants and four females. The focus group participants agreed that benefits of the S-STEM program fell in two categories, financial and motivational. They stated that without the scholarships they would have needed to work to complete school and felt that would have impacted their ability to be successful. The focus group participants said that the S-STEM scholarships gave them a sense of accomplishment, indicated to them that they were capable and that they could do more than they might otherwise have thought, and that the awards gave them motivation to attend classes, to stay “on top of things; classes and grades, and that the GPA requirement motivated them to invest time and effort in meeting that standard. The most valuable aspect(s) of the S-STEM project, as stated by the focus group participants, are: 1) the ability to continue as an undergraduate student in engineering or computer science major, 2) the ability to work toward a degree without incurring debt or, at least, with a limited amount of debt, 3) the meetings held each semester at which participants were reminded of project commitments and opportunities that were before them, 4) the opportunities presented to meet faculty members, and 5) being encouraged by project personnel. 

\section{Conclusion}
This paper presents the results and experiences of an NSF S-STEM grant that supported 39 undergraduate students in which there were 17 community college transfer students, and the rest were recruited high school seniors. Among the participants, there were 11 Hispanic students and 13 female students.  Seven students could not continue in the program as they could not maintain a GPA of at least 3.0. Altogether, 35 students have graduated successfully including three students who were dropped out from the S-STEM program due to low GPA. The most of the dropout students were found to be busy working long hours to support themselves financially. Increasing the scholarship amount could help these students spend more time in education than in work.  The overall performance of the community college transfer students was better than the students recruited from high school. All the attritions were from the group of the recruits from high school. Some better support strategies such as enrolling students in a learning community or preparatory summer courses in cohorts can help. Midterm mentoring was found to be helpful for students at risk to continue in the program. Several students who participated in undergraduate research were successful in obtaining employment in leading companies in the nation.

\begin{center}
\large \textbf {Acknowledgment}
\end{center}

This material is based upon work supported in part by the National Science Foundation under Grant DUE-1458096. Any opinions, findings, and conclusions or recommendations expressed in this material are those of the authors and do not necessarily reflect the views of the National Science Foundation.

\medskip

\bibliographystyle{plain}
\bibliography{155}

\end{document}