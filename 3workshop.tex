\documentclass{article}

%\input{preamble}
%\addbibresource{sample.bib}
%\title{Bloom's for Computing: Crafting Learning Outcomes with Enhanced Verb Lists for Computing Competencies\\
%	\title{Designing Learning Outcomes and Competencies using Bloom's for Computing\\
%	\vspace{0.2in}
%	\large{
%		Conference Tutorial
%}}

%\author{
%	D. Cenk Erdil\\
%	School of Computer Science {\&} Engineering\\
%	Sacred Heart University\\
%	Fairfield, CT 06825\\
%	\email{erdild@sacredheart.edu}
%}
\title{Designing Learning Outcomes and Competencies using Bloom's for Computing\footnote{\protect\input{copyleft}}
\\
\vspace{0.2in}
\large{
Conference Tutorial
}}


\author{
%	Christian Servin\affmark[1] and Markus Geissler\affmark[2] and Koudjo Koumadi\affmark[3] and \\ Pam Schmelz\affmark[4] and Cara Tang\affmark[5] and Cindy Tucker\affmark[6]\\
	Markus Geissler, Koudjo Koumadi, Pam Schmelz\\ Christian Servin, Cara Tang and Cindy Tucker \\
	Association of Computing Machinery\\
	Committee for Computing Education in\\ Community Colleges\\
%	
%	\affmark[1]El Paso Community College\\
%	El Paso, TX, \\ 
%	\email{cservin1@epcc.edu}\\
%	
%	\affmark[2]Cosumnes River College\\
%	Sacramento, CA\\
%	\email{geisslm@crc.losrios.edu}\\
%	
%	\affmark[3]Prince George's Community College\\
%	Largo, MD\\
%	\email{koumadkm@pgcc.edu}\\
%	
%	\affmark[4]Ivy Tech Community College\\
%	Columbus, IN\\
%	\email{pschmelz@ivytech.edu}\\
%	
%	\affmark[5]Portland Community College\\
%	Portland, OR\\
%	\email{cara.tang@pcc.edu}\\
%	
%	\affmark[6]Bluegrass Community and Technical College\\
%	Lexington, KY\\
%	\email{geisslm@crc.losrios.edu}\\
%	
}

\begin{document}
	\maketitle
	
	In this tutorial, participants will be introduced to Bloom's for Computing: Enhancing Bloom's Revised Taxonomy with Verbs for Computing Disciplines, a project of the ACM CCECC (Committee for Computing Education in Community Colleges)~\cite{Geissler2022,Servin2021,Tang2022,Tang2022a}. Due for final publication by the end of 2022, the Bloom's for Computing report offers a total 57 additional verbs across all six levels of Bloom's cognitive domain – Remembering, Understanding, Applying, Analyzing, Evaluating, Creating. The enhanced verb list is intended to support crafting more appropriate and less awkward learning outcomes and competencies that express the knowledge, skills, and dispositions required in computing disciplines. The Bloom's for Computing verb list and report is not just for use in future ACM curriculum guideline reports, but is primarily for educators in computing disciplines who find themselves needing to craft learning outcomes or competencies – whether for programs, courses, or individual modules; whether two-year, four-year, graduate, or K-12 level; whether faculty, instructional designers, or program coordinators.
	

	The presentation and activities in the proposed tutorial session are outlined below:
	\begin{enumerate}
		\item Introductions – tutorial facilitators \& participants
		\item Refresher on Bloom's Revised Taxonomy, its 6 cognitive levels, and common verbs lists
		\item Interactive discussion on how faculty approach writing learning outcomes and some of the challenges encountered
		\item Bloom's for Computing: Enhancing Bloom's Revised Taxonomy with Verbs for Computing Disciplines
		\begin{enumerate}
			\item Introduce the project \& the verbs
			\item Examples of learning outcomes using the Bloom's for Computing verbs
			\item Areas where the Bloom's for Computing verbs come in particularly handy
		\end{enumerate}
		\item Activity where participants write or modify learning outcomes for courses they teach
		\item Share out learning outcomes and thoughts on how the enhanced verbs might be used
		\item Wrap up
		
	\end{enumerate}
	

	\noindent Participants will be given a handout to take home with the complete list of verbs for each cognitive level.
	
	\noindent This tutorial is relevant for anyone involved in writing, revising, or updating learning outcomes for programs, courses, or instructional units in computing disciplines such as Computer Science, Information Technology, and Cybersecurity.
	
\section*{Biography}
Christian Servin is an Associate Professor of Computer Science at El Paso Community College; he has more than 15 years of experience teaching computing courses, developing best practices in computing education, and establishing academic and research partnerships between community colleges with ISDs and four-year colleges. Currently, he serves as a vice chair for the ACM Standing Committee for Computing Education in Community Colleges (CCECC), where he assists in the development and updates curricular guidelines in computing for 2-Year colleges and computing education projects. He serves as a member of several committees for The Texas Higher Education Coordinating Board representing Two-Year Colleges. He has served in various curricular guidelines recommendations such as the ACM Data Science Task Force and the CS2023: ACM/IEEE-CS/AAAI Computer Science Curricula

\medskip

 \bibliographystyle{plain}
\bibliography{3}

	%%	\printbibliography
%\bibliographystyle{plain}
%\small{
%	\bibliography{tutorial_abstract}
%}
\end{document}