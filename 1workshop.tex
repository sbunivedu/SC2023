\documentclass{article}

%\input{preamble}

%%\addbibresource{sample.bib}

\title{Programming Many-Core Architectures (GPUs) Using CUDA\footnote{\protect\input{copyleft}}
\\
\vspace{0.2in}
\large{
Conference Tutorial
}}

\author{
Eduardo Colmenares\\
Computer Science\\
Midwestern State University\\
Wichita Falls, Texas, 76308\\
\email{eduardo.colmenares@msutexas.edu}\\
}

\begin{document}
\maketitle

%%\begin{abstract}
\noindent Many-core devices also known as Graphical Processing Units (GPUs) have dominated the floating point race since 2013 and their benefits have been fueling the current artificial intelligence (AI), machine learning (ML), deep learning (DL), and data science (DS) revolution. Many of these fields benefit from the use of specialized frameworks, such as Tensor Flow, Keras, etc. Although extremely helpful for AI, ML, DL and DS related purposes, the use of these frameworks does not provide the user any knowledge about how to harness the potential of the underlying hardware for different applications or different fields to those mentioned above.
This tutorial intends to provide interested students and faculty a basic, but strong hands-on programming foundation regarding good practices and potential capabilities of modern GPUs. The tutorial focuses on GPU programming and not AI Frameworks.

%%\end{abstract}

\end{document}
